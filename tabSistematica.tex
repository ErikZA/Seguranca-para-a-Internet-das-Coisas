\begin{table}
\caption{Principais algoritmos de chave privada ou criptografia simétrica}
\label{tab1}
\begin{tabular}{|l|l|l|}
\hline
Algoritmo & Bits                                                     & Descrição                                                                                                                                                                                                                                                                                                                                                                                                                                                                                                                                                                                                                                                                                                                                                                                      \\ \hline
AES       & 128                                                      & \begin{tabular}[c]{@{}l@{}}O Advanced Encryption Standard (AES) é uma cifra de bloco, \\ anunciado pelo National Institute of Standards and Technology (NIST) \\ em  2003,  fruto  de  concurso  para  escolha  de  um  novo algoritmo  \\ de  chave  simétrica  para  proteger  informações  do governo federal, \\ sendo adotado como padrão pelo governo dos Estados  Unidos,  é  \\ um  dos  algoritmos  mais  populares,  desde 2006, usado   para   \\ criptografia   de   chave   simétrica,   sendo considerado como o \\ padrão substituto do DES. O AES tem um tamanho de bloco fixo em \\ 128 bits e uma chave com tamanho de 128, 192 ou 256 bits, ele é \\ rápido tanto em software quanto em hardware, é relativamente fácil \\ de executar e requer pouca memória.\end{tabular} \\ \hline
DES       & 56                                                       & \begin{tabular}[c]{@{}l@{}}O  Data  Encryption  Standard  (DES)  foi  o  algoritmo  simétrico mais  \\ disseminado  no mundo,  até  a  padronização  do  AES.  Foi criado  \\ pela  IBM  em  1977  e,  apesar  de  permitir  cerca  de  72 quadrilhões \\ de combinações, seu tamanho de chave (56 bits) é considerado pequeno, \\ tendo sido quebrado por "força bruta" em 1997 em um desafio lançado na \\ Internet. O NIST que lançou o desafio  mencionado,  recertificou  o  DES  \\ pela  última  vez  em 1993, passando então a recomendar o 3DES.\end{tabular}                                                                                                                                                                                                                         \\ \hline
3DES      & \begin{tabular}[c]{@{}l@{}}112 \\ ou \\ 168\end{tabular} & \begin{tabular}[c]{@{}l@{}}O 3DES é uma simples variação do DES, utilizando o em três ciframentos  \\ suscessivos,  podendo  empregar  uma  versão  com duas ou com três \\ chaves diferentes. É seguro, porém muito lento para ser um algoritmo \\ padrão.\end{tabular}                                                                                                                                                                                                                                                                                                                                                                                                                                                                                                                       \\ \hline
IDEA      & 128                                                      & \begin{tabular}[c]{@{}l@{}}O  International Data Encryption Algorithm (IDEA) foi  criado em  1991  \\ por  James  Massey  e  Xuejia  Lai  e  possui  patente  da suíça  ASCOM  \\ Systec.  O  algoritmo  é  estruturado  seguindo  as mesmas linhas  gerais do \\ DES. Mas    na   maioria    dos microprocessadores,  uma   implementação  \\ por   software   do IDEA  é  mais  rápida  do  que  uma  implementação  por  \\ software do  DES.  O  IDEA  é  utilizado  principalmente  no  mercado \\ financeiro  e  no  PGP,  o  programa  para  criptografia  de  e-mail pessoal \\ mais disseminado no mundo.\end{tabular}                                                                                                                                                                \\ \hline
\end{tabular}
\centering Fonte:\cite{STALLINGS:14}
\end{table}
